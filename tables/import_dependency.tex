% CO2 Emissions Table
\begin{columns}[T]
    \begin{column}{0.6\textwidth}
        \begin{table}[h]
            \footnotesize
            \begin{tabular}{lc}
                \toprule
                \textbf{Kraj} & \textbf{Cena energii (PPS)} \\
                \midrule
                Belgia & 76.1\% \\
                Holandia & 70.44\% \\
                Portugalia & 66.87\% \\
                Niemcy & 66.38\% \\
                Słowacja & 57.73\% \\
                Słowenia & 49.27\% \\
                \textcolor{renewable}{\textbf{Poland}} & \textcolor{renewable}{\textbf{48.02\%}} \\
                Francja & 44.87\% \\
                Czechy & 41.68\% \\
                \rowcolor{green!10}
                Szwecja & 26.39\% \\
                \rowcolor{green!10}
                Estonia & 3.47\% \\
                \bottomrule
            \end{tabular}
                    \vspace{-0.1em}
            \caption{Procent wykorzystywanej pochodzącej z importu (2023)}
        \end{table}
    \end{column}

    \begin{column}{0.5\textwidth}
        \vspace{-2.2em}
        \begin{tcolorbox}[colback=blue!10,colframe=blue!50!black,title=Wniosek kluczowy]
            Stopień zależności energetycznej Polski od importu jest umiarkowany.
        \end{tcolorbox}
        \vspace{-0em}
        \begin{itemize}
            \item \small Szwecja i Estonia, posiadające bardzo wysoki udział źródeł odnawialnych, posiadają najniższe współczynniki zależności energetycznej.
            \item \small Istnieje silna zależność ujemna między udziałem OZE a stopniem zależności energetycznej krajów Europy.
        \end{itemize}
    \end{column}
\end{columns}
