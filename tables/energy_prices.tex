% CO2 Emissions Table
\begin{columns}[T]
    \begin{column}{0.6\textwidth}
        \begin{table}[h]
            \footnotesize
            \begin{tabular}{lc}
                \toprule
                \textbf{Kraj} & \textbf{Cena energii (PPS)} \\
                \midrule
                Czechy & 0.42 \\
                Niemcy & 0.35 \\
                \textcolor{renewable}{\textbf{Poland}} & \textcolor{renewable}{\textbf{0.32}} \\
                Portugalia & 0.31 \\
                Belgia & 0.30 \\
                Francja & 0.26 \\
                Estonia & 0.25 \\
                Słowenia & 0.24 \\
                \rowcolor{green!10}
                Słowacja & 0.22 \\
                \rowcolor{green!10}
                Holandia & 0.21 \\
                \rowcolor{green!10}
                Szwecja & 0.20 \\
                \bottomrule
            \end{tabular}
                    \vspace{-0.1em}
            \caption{Ceny energii dla gospodarstw domowych (PPS, uwzględniając podatki, 2023)}
        \end{table}
    \end{column}

    \begin{column}{0.4\textwidth}
        \vspace{-2.2em}
        \begin{tcolorbox}[colback=purple!10,colframe=purple!50!black,title=Wniosek kluczowy]
            Ceny za energię w Polsce są jednymi z wyższych w Europie.
        \end{tcolorbox}
        \vspace{-0em}
        \begin{itemize}
            \item \small Kraje z wysokim sumarycznym udziałem procentowym źródeł odnawialnych i energii nuklearnej w strukturze produkcji energii zwykle cieszą się relatywnie niższymi cenami energii.
        \end{itemize}
    \end{column}
\end{columns}
