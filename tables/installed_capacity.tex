% Installed Capacity Table
\begin{table}
    \centering
    \footnotesize % Reduce font size to make table fit
    \setlength{\tabcolsep}{5pt} % Reduce column spacing
    \begin{tabular}{lcr}
        \toprule
        \textbf{Typ elektrowni} & \textbf{Moc [MW]} & \textbf{Udział [\%]} \\
        \midrule
        Elektrownie cieplne oparte na węglu kamiennym & 23,711 & 32.3 \\
        Elektrownie cieplne oparte na węglu brunatnym & 8,249 & 11.2 \\
        Elektrownie gazowe & 5,976 & 8.1 \\
        Elektrownie wodne & 2,430 & 3.3 \\
        \textcolor{renewable}{\textbf{Elektrownie wiatrowe i inne odnawialne}} & 31,823 & 43.2 \\
        \midrule
        \textbf{Razem} & \textbf{73,489} & \textbf{100} \\
        \bottomrule
    \end{tabular}
    \caption{Moc zainstalowana w podziale na typ elektrowni}
\end{table}
\vspace{-0.5em} % Reduce vertical space between table and list
    \begin{itemize}
        \item Farmy wiatrowe oraz instalacje fotowoltaiczne aktualnie stanowią 43.2\% mocy zainstalowanej w polskim systemie elektroenergetycznym.
        \item Razem stanowią typ elektrowni o największym udziale procentowym w Polsce.
    \end{itemize}
