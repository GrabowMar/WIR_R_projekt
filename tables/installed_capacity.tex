% Compact Installed Capacity Table - Version 1
\begin{table}
    \centering
    \tiny % Even smaller font
    \setlength{\tabcolsep}{3pt} % Further reduce column spacing
    \begin{tabular}{>{\raggedright\arraybackslash}p{4.2cm}rr}
        \toprule
        \textbf{Typ elektrowni} & \textbf{Moc [MW]} & \textbf{Udział [\%]} \\
        \midrule
        Węgiel kamienny & 23,711 & 32.3 \\
        Węgiel brunatny & 8,249 & 11.2 \\
        Elektrownie gazowe & 5,976 & 8.1 \\
        Elektrownie wodne & 2,430 & 3.3 \\
        \textcolor{renewable}{\textbf{OZE (wiatr + inne)}} & \textcolor{renewable}{\textbf{31,823}} & \textcolor{renewable}{\textbf{43.2}} \\
        \midrule
        \textbf{Razem} & \textbf{73,489} & \textbf{100.0} \\
        \bottomrule
    \end{tabular}
    \caption{\footnotesize Moc zainstalowana według typu elektrowni}
\end{table}


\begin{itemize}[leftmargin=*, itemsep=0.1em]
    \item[\footnotesize\faIcon{leaf}] {\footnotesize OZE stanowią \textbf{43.2\%} całkowitej mocy zainstalowanej}
    \item[\footnotesize\faIcon{chart-line}] {\footnotesize Największy udział spośród wszystkich typów elektrowni}
\end{itemize}