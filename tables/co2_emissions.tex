% CO2 Emissions Table
\begin{columns}[T]
    \begin{column}{0.6\textwidth}
        \begin{table}[h]
            \footnotesize
            \begin{tabular}{lc}
                \toprule
                \textbf{Kraj} & \textbf{CO\textsubscript{2} per capita} \\
                \midrule
                Estonia & 8.87 \\
                Czechy & 8.52 \\
                \textcolor{renewable}{\textbf{Poland}} & \textcolor{renewable}{\textbf{7.63}} \\
                Belgia & 7.18 \\
                Niemcy & 7.06 \\
                Holandia & 7.09 \\
                Słowenia & 6.81 \\
                Słowacja & 6.4 \\
                Francja i Monako & 4.25 \\
                \rowcolor{green!10}
                Szwecja & 3.43 \\
                \rowcolor{green!10}
                Portugalia & 3.70 \\
                \bottomrule
            \end{tabular}
            \caption{Emisje CO\textsubscript{2} wybranych krajów europejskich (2023)}
        \end{table}
    \end{column}

    \begin{column}{0.4\textwidth}
        \vspace{-2.2em}
        \begin{tcolorbox}[colback=green!10,colframe=green!50!black,title=Wniosek kluczowy]
            Polska osiąga jeden z gorszych wyników wśród krajów europejskich.
        \end{tcolorbox}
        \vspace{-0em}
        \begin{itemize}
            \item \small Kraje o wysokim sumarycznym udziale źródeł odnawialnych i nuklearnych produkują mniej CO\textsubscript{2} na osobę.
            \item Kraje europy środkowo-wschodniej uzyskują gorsze wyniki niż europy zachodniej.
        \end{itemize}
    \end{column}
\end{columns}
