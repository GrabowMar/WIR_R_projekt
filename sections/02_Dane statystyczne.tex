\section{Dane statystyczne}


\begin{frame}{Główne źródła danych}
    \begin{center}
        \begin{tikzpicture}
            % PSE Node
            \node[draw, rounded corners, fill=green!10, text width=3.5cm, align=center, minimum height=2.5cm] (pse) at (-4,0) {
                \textbf{PSE}\\
                \small{Polskie Sieci Elektroenergetyczne}\\[0.4cm]
                \footnotesize{Państwowy operator}\\
                \footnotesize{systemu przesyłowego}\\
                \footnotesize{w Polsce}
            };
            
            % EDGAR Node
            \node[draw, rounded corners, fill=blue!10, text width=3.5cm, align=center, minimum height=2.5cm] (edgar) at (0,0) {
                \textbf{EDGAR}\\
                \small{European Commission}\\[0.4cm]
                \footnotesize{Globalna baza danych}\\
                \footnotesize{emisji atmosferycznych}\\
                \footnotesize{JRC EC}
            };
            
            % Eurostat Node
            \node[draw, rounded corners, fill=red!10, text width=3.5cm, align=center, minimum height=2.5cm] (eurostat) at (4,0) {
                \textbf{Eurostat}\\
                \small{Energy Database}\\[0.4cm]
                \footnotesize{Oficjalny urząd}\\
                \footnotesize{statystyczny}\\
                \footnotesize{Unii Europejskiej}
            };
        \end{tikzpicture}
    \end{center}
    
    \vspace{0.5cm}
    
    \begin{itemize}
        \item Wszystkie źródła zapewniają oficjalne, zweryfikowane dane instytucjonalne
        \item Dane pochodzą z różnych poziomów: krajowego, europejskiego i globalnego
    \end{itemize}
\end{frame}


\begin{frame}{Zakres i charakterystyka danych}
    \begin{columns}[T]
        \begin{column}{0.48\textwidth}
            \begin{beamercolorbox}[rounded=true, shadow=false, sep=10pt]{block body}
                \textbf{\large Źródła}
                \vspace{0.3cm}
                
                    \begin{tabular}{@{}>{\color{renewable}}l @{\hspace{0.3cm}} p{10cm}@{}}
                            \faIcon{industry} & \textbf{PSE} - struktura energetyczna \newline kraju \\[0.3cm]
                            \faIcon{globe} & \textbf{EDGAR} - emisje CO\textsubscript{2} \newline per capita \\[0.3cm]
                            \faIcon{chart-line} & \textbf{Eurostat} - ceny energii, \newline udział OZE i   energii \newline nuklearnej, udział importów \newline w wykorzystanej energii \newline państw europejskich \\[0.3cm]
                    \end{tabular}

            \end{beamercolorbox}
        \end{column}
        
\begin{column}{0.48\textwidth}
    \begin{beamercolorbox}[rounded=true, shadow=false, sep=5pt]{block body}
        \textbf{\large Parametry Analizy}
        \vspace{0.1cm}
        
        \begin{tikzpicture}
            \node[draw=none, fill=renewable!10, rounded corners, 
                  text width=\linewidth-20pt, align=center, anchor=north] (box1) at (0,0) {
                \begin{tabular}{@{}c@{}}
                    \textbf{23} przeanalizowane raporty
                \end{tabular}
            };

            \node[draw=none, fill=renewable!10, rounded corners, 
                  text width=\linewidth-20pt, align=center, anchor=north] (box2) at (0,-1.) {
                \begin{tabular}{@{}c@{}}
                    \textbf{27} krajów europejskich
                \end{tabular}
            };

            \node[draw=none, fill=renewable!10, rounded corners, 
                  text width=\linewidth-20pt, align=center, anchor=north] (box3) at (0,-2) {
                \begin{tabular}{@{}c@{}}
                    \textbf{2004--2024} zakres czasowy
                \end{tabular}
            };

            \node[draw=none, fill=renewable!10, rounded corners, 
                  text width=\linewidth-20pt, align=center, anchor=north] (box4) at (0,-3) {
                \begin{tabular}{@{}c@{}}
                    \textbf{284} przeanalizowane rekordy
                \end{tabular}
            };

            \node[draw=none, fill=renewable!10, rounded corners, 
                  text width=\linewidth-20pt, align=center, anchor=north] (box5) at (0,-4) {
                \begin{tabular}{@{}c@{}}
                    \textbf{10} wykorzystanych cech
                \end{tabular}
            };
        \end{tikzpicture}
    \end{beamercolorbox}
\end{column}

    \end{columns}
\end{frame}