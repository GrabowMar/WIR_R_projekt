% ==============================================================================
% WAŻNE: Aby cienie działały poprawnie, upewnij się, że biblioteka 'shadows'
% jest dodana do pliku 'config/packages.tex'.
%
% Znajdź tę linię w 'config/packages.tex':
% \usetikzlibrary{shapes,arrows,positioning,patterns,calc,decorations.pathreplacing}
% 
% I dodaj do niej 'shadows', w ten sposób:
% \usetikzlibrary{shapes,arrows,positioning,patterns,calc,decorations.pathreplacing,shadows}
% ==============================================================================

% Wykorzystane narzędzia i biblioteki
\section{Wykorzystane narzędzia i biblioteki}

\begin{frame}{Metodologia pracy z danymi}
    \centering
    \begin{tikzpicture}[
        % POPRAWKA: Użycie node distance jako domyślnej odległości upraszcza kod
        % i czyni go bardziej odpornym na błędy parsowania.
        node distance=0.7cm,
        step/.style={
            rectangle, 
            rounded corners, 
            draw=coalplant, 
            fill=renewable!10, 
            text width=0.8\textwidth, 
            minimum height=1.5cm, 
            text centered,
            drop shadow
        },
        arrow/.style={
            ->, 
            ultra thick, 
            color=renewable,
            shorten >=2pt,
            shorten <=2pt
        }
    ]
        % Zdefiniowanie kroków metodologii
        \node (step1) [step] {\highlight{1. Pozyskanie danych} \\ \small z oficjalnych źródeł europejskich i krajowych};
        % POPRAWKA: Użycie składni 'below=of step1' jest bardziej jednoznaczne
        % i stabilniejsze niż jawne podawanie odległości przy każdym węźle.
        \node (step2) [step, below=of step1] {\highlight{2. Przetwarzanie i analiza} \\ \small za pomocą narzędzi programistycznych w języku Python};
        \node (step3) [step, below=of step2] {\highlight{3. Wizualizacja wyników} \\ \small w postaci wykresów, map i tabel porównawczych};
        \node (step4) [step, below=of step3] {\highlight{4. Prezentacja wyników} \\ \small w profesjonalnym układzie przy użyciu LaTeX i Beamer};

        % Połączenie kroków strzałkami
        \draw [arrow] (step1.south) -- (step2.north);
        \draw [arrow] (step2.south) -- (step3.north);
        \draw [arrow] (step3.south) -- (step4.north);
    \end{tikzpicture}
\end{frame}

\begin{frame}{Główne narzędzia analizy danych}
    % POPRAWKA: Zmniejszenie szerokości kolumn (np. do 0.49\textwidth)
    % daje więcej przestrzeni i może zapobiec błędom 'Overfull hbox'.
    \begin{columns}[T, totalwidth=\textwidth]
        \begin{column}{0.49\textwidth}
            \begin{beamercolorbox}[rounded=true, shadow=true, sep=10pt]{block body}
                \textbf{\large \faIcon{python}\ Python -- Analiza Danych}
                \vspace{0.3cm}
                \begin{itemize}
                    \item[\faIcon{table}] \textbf{pandas} -- manipulacja i analiza danych tabelarycznych
                    \item[\faIcon{calculator}] \textbf{numpy} -- zaawansowane obliczenia numeryczne
                    \item[\faIcon{chart-bar}] \textbf{matplotlib} -- tworzenie statycznych i interaktywnych wizualizacji
                    \item[\faIcon{globe-europe}] \textbf{cartopy} -- generowanie map i projekcji geograficznych
                \end{itemize}
            \end{beamercolorbox}
        \end{column}
        
        \begin{column}{0.49\textwidth}
            \begin{beamercolorbox}[rounded=true, shadow=true, sep=10pt]{block body}
                \textbf{\large \faIcon{file-code}\ LaTeX -- Skład Dokumentów}
                \vspace{0.3cm}
                \begin{itemize}
                    \item[\faIcon{desktop}] \textbf{beamer} -- tworzenie profesjonalnych prezentacji
                    \item[\faIcon{pen-fancy}] \textbf{tikz} -- rysowanie diagramów i grafiki wektorowej
                    \item[\faIcon{th-list}] \textbf{booktabs} -- skład estetycznych i czytelnych tabel
                    \item[\faIcon{palette}] \textbf{motyw Argüelles} -- nowoczesny i minimalistyczny design
                \end{itemize}
            \end{beamercolorbox}
        \end{column}
    \end{columns}
\end{frame}

\begin{frame}{Przepływ pracy -- od danych do prezentacji}
    \begin{center}
        \begin{tikzpicture}[
            node distance=0.2cm,
            block/.style={
                rectangle, rounded corners, draw=coalplant,
                text width=2.8cm, align=center, minimum height=1.5cm,
                font=\bfseries, drop shadow
            },
            arrow/.style={
                ->, ultra thick, shorten >=3pt, shorten <=3pt,
                color=renewable
            }
        ]
            % Zdefiniowanie etapów przepływu pracy
            \node[block, fill=renewable!20] (data) at (0,0) {
                Źródła Danych \\ \footnotesize\normalfont Eurostat, PSE, EDGAR
            };
            % POPRAWKA: Zmniejszenie odległości między blokami, aby cały diagram
            % zmieścił się w szerokości slajdu.
            \node[block, fill=gasplant!20, right=0.6cm of data] (process) {
                Przetwarzanie \\ \footnotesize\normalfont Python, pandas
            };
            \node[block, fill=solar!30, right=0.6cm of process] (viz) {
                Wizualizacja \\ \footnotesize\normalfont matplotlib, cartopy
            };
            \node[block, fill=coalplant!20, right=0.6cm of viz] (output) {
                Prezentacja \\ \footnotesize\normalfont LaTeX, beamer
            };

            % Narysowanie strzałek między etapami
            \draw[arrow] (data) -- (process);
            \draw[arrow] (process) -- (viz);
            \draw[arrow] (viz) -- (output);
            
            % Dodanie etykiet pod każdym etapem
            \node[below=0.3cm of data, font=\small\itshape, text=coalplant] {CSV, JSON};
            \node[below=0.3cm of process, font=\small\itshape, text=coalplant] {Analiza, czyszczenie};
            \node[below=0.3cm of viz, font=\small\itshape, text=coalplant] {Wykresy, mapy};
            \node[below=0.3cm of output, font=\small\itshape, text=coalplant] {PDF};
        \end{tikzpicture}
    \end{center}
    
    \vspace{0.5cm}
    
    \begin{itemize}
        \item Proces w pełni \highlight{automatyczny i reprodukowalny}
        \item Kontrola wersji za pomocą \highlight{Git i GitHub}
        \item Współpraca zespołowa przez \highlight{Overleaf}
    \end{itemize}
\end{frame}

\begin{frame}{Wykorzystane biblioteki Python}

    \begin{table}
        \rowcolors{2}{coalplant!10}{white}

        \begin{tabularx}{\textwidth}{l l >{\raggedright\arraybackslash}X}
            \toprule
            \textbf{Biblioteka} & \textbf{Wersja} & \textbf{Zastosowanie} \\
            \midrule
            pandas & 2.0+ & Manipulacja danymi tabelarycznymi, analiza szeregów czasowych \\
            numpy & 1.24+ & Obliczenia numeryczne, operacje na macierzach \\
            matplotlib & 3.7+ & Tworzenie wykresów słupkowych, liniowych i kołowych \\
            cartopy & 0.21+ & Tworzenie map geograficznych i projekcji kartograficznych \\
            scipy & 1.10+ & Zaawansowana analiza statystyczna i testy korelacji \\
            pathlib & -- & Zarządzanie ścieżkami plików w sposób obiektowy \\
            \bottomrule
        \end{tabularx}
        \caption{Kluczowe biblioteki Python wykorzystane w projekcie}
    \end{table}
Wszystkie biblioteki w \highlight{aktualnych wersjach stabilnych}
\end{frame}


\begin{frame}{Środowisko pracy i narzędzia wspomagające}
    \begin{table}[h]
        \centering
        \rowcolors{2}{coalplant!10}{white}
        % POPRAWKA: Znacząco zmniejszono szerokość kolumny, aby tabela
        % na pewno zmieściła się w ramce.
        \begin{tabular}{l >{\raggedright\arraybackslash}p{7.5cm}}
            \toprule
            \textbf{Kategoria} & \textbf{Narzędzia} \\
            \midrule
            \faIcon{code}\ Edytory kodu & VS Code, Jupyter Lab, PyCharm \\
            \faIcon{code-branch}\ Kontrola wersji & Git, GitHub \\
            \faIcon{users}\ Współpraca & Overleaf (LaTeX), Google Drive \\
            \faIcon{python}\ Środowisko Python & conda, pip, virtual environments \\
            \faIcon{robot}\ Asystenci AI & Claude, GitHub Copilot \\
            \faIcon{vial}\ Testowanie & pytest, unittest \\
            \bottomrule
        \end{tabular}
    \end{table}
    
    \vspace{0.5cm}
    
    \begin{itemize}
        \item \highlight{Overleaf} umożliwił efektywną współpracę przy tworzeniu prezentacji LaTeX.
        \item \highlight{Asystenci AI} znacząco przyspieszyli proces programowania i debugowania.
        \item \highlight{GitHub} zapewnił pełną kontrolę wersji oraz transparentność projektu.
    \end{itemize}
\end{frame}

\begin{frame}{Kluczowy skrypt Python: Analiza i wizualizacja danych}
    
    \textbf{Cel i metodologia}
    \begin{itemize}
        \item \highlight{Automatyzacja procesu}: Od wczytania danych za pomocą \textbf{pandas}, po generowanie map w \textbf{matplotlib} i \textbf{cartopy}.
        \item \highlight{Analiza dynamiki}: Obliczanie procentowych zmian w produkcji energii (całkowitej i OZE) w zadanym okresie.
        \item \highlight{Wizualizacja wyników}: Tworzenie czytelnych map choropletowych Europy, które przedstawiają trendy energetyczne.
    \end{itemize}

\end{frame}