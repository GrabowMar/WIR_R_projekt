% Wykorzystane narzędzia i biblioteki
\section{Wykorzystane narzędzia i biblioteki}

\begin{frame}{Metodologia pracy z danymi}
    \begin{enumerate}
        \item \highlight{Pozyskanie danych} z oficjalnych źródeł europejskich i krajowych
        \vspace{0.3em}
        
        \item \highlight{Przetwarzanie i analiza} za pomocą narzędzi programistycznych Python
        \vspace{0.3em}
        
        \item \highlight{Wizualizacja wyników} w postaci wykresów, map i tabel porównawczych
        \vspace{0.3em}
        
        \item \highlight{Prezentacja wyników} w profesjonalnym układzie LaTeX
    \end{enumerate}
\end{frame}

\begin{frame}{Główne narzędzia analizy danych}
    \begin{columns}[T]
        \begin{column}{0.48\textwidth}
            \begin{beamercolorbox}[rounded=true, shadow=false, sep=10pt]{block body}
                \textbf{\large Python - analiza danych}
                \vspace{0.3cm}
                
                \begin{itemize}
                    \item \textbf{pandas} -- manipulacja danych tabelarycznych
                    \item \textbf{numpy} -- obliczenia numeryczne i statystyczne
                    \item \textbf{matplotlib} -- tworzenie wykresów i wizualizacji
                    \item \textbf{cartopy} -- mapy geograficzne i projekcje
                \end{itemize}
            \end{beamercolorbox}
        \end{column}
        
        \begin{column}{0.48\textwidth}
            \begin{beamercolorbox}[rounded=true, shadow=false, sep=10pt]{block body}
                \textbf{\large LaTeX - skład dokumentów}
                \vspace{0.3cm}
                
                \begin{itemize}
                    \item \textbf{beamer} -- prezentacje naukowe
                    \item \textbf{tikz} -- diagramy i schematy
                    \item \textbf{booktabs} -- profesjonalne tabele
                    \item \textbf{motyw Argüelles} -- nowoczesny design
                \end{itemize}
            \end{beamercolorbox}
        \end{column}
    \end{columns}
\end{frame}

\begin{frame}{Przepływ pracy -- od danych do prezentacji}
    \begin{center}
        \begin{tikzpicture}
            % Data sources
            \node[draw, rounded corners, fill=renewable!20, text width=2.5cm, align=center, minimum height=1.2cm] (data) at (0,0) {
                \textbf{Źródła danych}\\
                \footnotesize Eurostat, PSE, EDGAR
            };
            
            % Processing
            \node[draw, rounded corners, fill=blue!20, text width=2.5cm, align=center, minimum height=1.2cm] (process) at (4,0) {
                \textbf{Przetwarzanie}\\
                \footnotesize Python, pandas
            };
            
            % Visualization
            \node[draw, rounded corners, fill=orange!20, text width=2.5cm, align=center, minimum height=1.2cm] (viz) at (8,0) {
                \textbf{Wizualizacja}\\
                \footnotesize matplotlib, cartopy
            };
            
            % Output
            \node[draw, rounded corners, fill=purple!20, text width=2.5cm, align=center, minimum height=1.2cm] (output) at (12,0) {
                \textbf{Prezentacja}\\
                \footnotesize LaTeX, beamer
            };
            
            % Arrows
            \draw[->, thick, renewable] (data) -- (process);
            \draw[->, thick, renewable] (process) -- (viz);
            \draw[->, thick, renewable] (viz) -- (output);
            
            % Labels
            \node[below=0.5cm of data, font=\tiny] {CSV, JSON};
            \node[below=0.5cm of process, font=\tiny] {analiza, czyszczenie};
            \node[below=0.5cm of viz, font=\tiny] {wykresy, mapy};
            \node[below=0.5cm of output, font=\tiny] {PDF};
        \end{tikzpicture}
    \end{center}
    
    \vspace{0.5cm}
    
    \begin{itemize}
        \item Proces w pełni automatyczny i reprodukowalny
        \item Kontrola wersji za pomocą Git i GitHub
        \item Współpraca zespołowa przez Overleaf
    \end{itemize}
\end{frame}

\begin{frame}{Wykorzystane biblioteki Python}
    \begin{table}
        \centering
        \footnotesize
        \begin{tabular}{llp{6cm}}
            \toprule
            \textbf{Biblioteka} & \textbf{Wersja} & \textbf{Zastosowanie} \\
            \midrule
            pandas & 2.0+ & Manipulacja danymi tabelarycznymi, analiza szeregów czasowych \\
            numpy & 1.24+ & Obliczenia numeryczne, operacje na macierzach \\
            matplotlib & 3.7+ & Tworzenie wykresów słupkowych, liniowych i kołowych \\
            cartopy & 0.21+ & Mapy geograficzne, projekcje kartograficzne \\
            scipy & 1.10+ & Analiza statystyczna, testy korelacji \\
            pathlib & -- & Zarządzanie ścieżkami plików \\
            \bottomrule
        \end{tabular}
        \caption{Kluczowe biblioteki wykorzystane w projekcie}
    \end{table}
    
    \vspace{0.3cm}
    
    \begin{itemize}
        \item Wszystkie biblioteki w aktualnych wersjach stabilnych
        \item Kompatybilność z Python 3.9+
    \end{itemize}
\end{frame}

\begin{frame}{Środowisko pracy i narzędzia wspomagające}
    \begin{table}[h]
        \footnotesize
        \begin{tabular}{lp{8cm}}
            \toprule
            \textbf{Kategoria} & \textbf{Narzędzia} \\
            \midrule
            Edytory kodu & VS Code, Jupyter Lab, PyCharm \\
            Kontrola wersji & Git, GitHub \\
            Współpraca & Overleaf (LaTeX), Google Drive \\
            Środowisko Python & conda, pip, virtual environments \\
            Asystenci AI & Claude \\
            Testowanie & pytest, unittest \\
            \bottomrule
        \end{tabular}
    \end{table}
    
    \vspace{0.5cm}
    
    \begin{itemize}
        \item \highlight{Overleaf} umożliwił efektywną współpracę przy tworzeniu prezentacji
        \item \highlight{Asystenci AI} przyspieszyły proces programowania i debugowania
        \item \highlight{GitHub} zapewnił pełną kontrolę wersji kodu i danych
    \end{itemize}
\end{frame}

\begin{frame}{Struktura projektu}
    \begin{columns}[T]
        \begin{column}{0.6\textwidth}
            \begin{itemize}
                \item \textbf{data/} -- surowe dane z Eurostat, PSE, EDGAR
                \item \textbf{scripts/} -- skrypty Python do analizy
                \item \textbf{outputs/} -- wygenerowane wykresy i tabele
                \item \textbf{presentation/} -- kod LaTeX prezentacji
                \item \textbf{docs/} -- dokumentacja metodologii
            \end{itemize}
        \end{column}

        \begin{column}{0.4\textwidth}
            \begin{beamercolorbox}[rounded=true, shadow=false, sep=8pt]{block body}
                \textbf{Statystyki projektu:}
                \vspace{0.2cm}
                
                \begin{tabular}{lr}
                    Pliki Python: & 12 \\
                    Linie kodu: & 1,800+ \\
                    Wykresy: & 15 \\
                    Tabele: & 8 \\
                    Slajdy: & 25+ \\
                \end{tabular}
            \end{beamercolorbox}
        \end{column}
    \end{columns}
    
    \vspace{0.3cm}
    
    \begin{itemize}
        \item Projekt w pełni \highlight{reprodukowalny} -- każdy wynik można odtworzyć
        \item Kod dostępny w repozytorium GitHub z pełną dokumentacją
    \end{itemize}
\end{frame}